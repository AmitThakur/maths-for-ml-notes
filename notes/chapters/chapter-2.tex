\chapter{Probability}

\section{Sample Space and Probability}

\subsection{Sets}

\begin{gather*}
    S_1 = \{x_1, x_2, \dots, x_n\}\\
    S_2 = \{x_1, x_2, \dots\}\\
    S_3 = \{x \mid x \text{ satisfies } P\}\\
    S_4 = \{x \mid 3 \leq x \leq 5, x \in \mathbb{R} \}\\
\end{gather*}

\begin{itemize}
    \item $x_2 \in S_1$.
    \item $S_2$ is an countably infinite set, as elements are enumerable.
    \item $S_4$ is
\end{itemize}

Now let's consider following sets:
\begin{gather*}
    S_1 = \{x_1, x_2, \dots, x_n\}\\
    S_2 = \{x_1, x_2\}\\
    S_3 = \{x_2, x_1\}\\
\end{gather*}

\begin{itemize}
    \item $S_2 \subset S_1$
    \item $S_2 = S_3$
    \item $\Omega$: A universal set
\end{itemize}

\subsection{Set Operations}

\begin{itemize}
    \item Complement: $S^c = \{x \in \Omega \mid x \notin S\}$
    \item $\Omega^c = \varnothing \text{ (empy set)}$
\end{itemize}